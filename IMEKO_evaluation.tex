%%%%%%%%%%%%%%%%%%%%%%%%%%%%%%%%%%%%%%%%%
% Engineering Calculation Paper
% LaTeX Template
% Version 1.0 (20/1/13)
% This template has been downloaded from:
% http://www.LaTeXTemplates.com
% Original author:
% Dmitry Volynkin (dim_voly@yahoo.com.au)
% Adapted by:
% Rafael Celestre
% License:
% CC BY-NC-SA 3.0 (http://creativecommons.org/licenses/by-nc-sa/3.0/)
%%%%%%%%%%%%%%%%%%%%%%%%%%%%%%%%%%%%%%%%%

%----------------------------------------------------------------------------------------
%	PACKAGES AND OTHER DOCUMENT CONFIGURATIONS  - DO NOT ALTER
%----------------------------------------------------------------------------------------

\documentclass[12pt,a4paper]{article} % Use A4 paper with a 12pt font size - different paper sizes will require manual recalculation of page margins and border positions

\usepackage{marginnote} % Required for margin notes
\usepackage{wallpaper} % Required to set each page to have a background
\usepackage{lastpage} % Required to print the total number of pages
\usepackage[left=1.3cm,right=4.6cm,top=1.8cm,bottom=4.0cm,marginparwidth=3.4cm]{geometry} % Adjust page margins
\usepackage{amsmath} % Required for equation customization
\usepackage{amssymb} % Required to include mathematical symbols
\usepackage{xcolor} % Required to specify colors by name

\usepackage{fancyhdr} % Required to customize headers
\setlength{\headheight}{80pt} % Increase the size of the header to accommodate meta-information
\pagestyle{fancy}\fancyhf{} % Use the custom header specified below
\renewcommand{\headrulewidth}{0pt} % Remove the default horizontal rule under the header

\setlength{\parindent}{0cm} % Remove paragraph indentation
\newcommand{\tab}{\hspace*{2em}} % Defines a new command for some horizontal space

\newcommand\BackgroundStructure{ % Command to specify the background of each page
\setlength{\unitlength}{1mm} % Set the unit length to millimeters

\setlength\fboxsep{0mm} % Adjusts the distance between the frameboxes and the borderlines
\setlength\fboxrule{0.5mm} % Increase the thickness of the border line
\put(10, 10){\fcolorbox{black}{blue!1}{\framebox(155,247){}}} % Main content box
\put(165, 10){\fcolorbox{black}{blue!3}{\framebox(37,247){}}} % Margin box
\put(10, 262){\fcolorbox{black}{white!10}{\framebox(192, 25){}}} % Header box
\put(137, 263){\includegraphics[height=23mm,keepaspectratio]{imeko_logo.png}} % Logo box - maximum height/width: 
}

%----------------------------------------------------------------------------------------
%	HEADER INFORMATION - complete with your own info!!!
%----------------------------------------------------------------------------------------
\fancyhead[L]{
\textbf{Manuscript evaluation form} \quad\quad\quad \textbf{page} \thepage/\pageref{LastPage}\\
\textbf{Abstract number}: \#69                          \\ %don't remove line breakers
\textbf{First author's name}: John Doe                  \\ %don't remove line breakers
\textbf{Reviewer}: reviewer's name
}
%----------------------------------------------------------------------------------------
\begin{document}
\AddToShipoutPicture{\BackgroundStructure}
%----------------------------------------------------------------------------------------
%	Evaluation form
%----------------------------------------------------------------------------------------

%================= General information on the paper
\section{Paper title} 
Here a very long paper title that you will copy and paste from the abstract you are evaluating.

\subsection*{General information}

\marginnote{\# pages}   % insert here the number of pages
\textbf{Number of pages}

\marginnote{\# references} % insert here the number of references
\textbf{Number of references}

%================= Detailed evaluation
\section{Evaluation criteria}
To each item bellow a grade from 0 to 5 is given, where 0 is the lowest score and 5, the highest.\\

\marginnote{\# grade} % grade related to the item below
1.) Does the article fit into the topic of the TC1-TC2 symposium?
\\

\marginnote{\# grade} % grade related to the item below
2.) Does the paper have a practical, innovative or didactic value?
\\

\marginnote{\# grade} % grade related to the item below
3.) Is the statement of the problem clear?
\\

\marginnote{\# grade} % grade related to the item below
4.) Does the abstract match what is presented on the article?
\\

\marginnote{\# grade} % grade related to the item below
5.) Is the methodology adequate/correct/sound?
\\

\marginnote{\# grade} % grade related to the item below
6.) Is the conclusion of the article pertinent to what is presented?
\\

\marginnote{\# grade} % grade related to the item below
7.) Are the references cited correctly?
\\

\marginnote{\# grade} % grade related to the item below
8.) Are language and register adequate?
\\

\marginnote{\# grade} % grade related to the item below
9.) Does the article follow the template for the conference?

%================= Final decision
\section{Paper evaluation}

% To select one of the categories, comment the bold faced one and comment. Comments begin with % 

From the following categories: \\

A) Accept unconditionally;                              \\
%\textbf{A) Accept unconditionally;}                     \\
B) Accept with minor revision without further review;   \\
%\textbf{B) Accept with minor revision without further review;}   \\
C) Accept with major revision and further review;       \\
%\textbf{C) Accept with major revision and further review;}        \\
D) Reject.
%\textbf{D) Reject.}
\\

To be presented as:\\

A) Oral presentation; \\
%\textbf{A) Oral presentation};
B) Poster presentation.
%\textbf{B) Poster presentation.}
%\subsection*{Reason for refusal (if applicable)}

%================= Space reserved for inserting some comments to the authors
\section{General comments and suggestions}

\section{Reviewer's conflicts of interest}
% list of affiliations from the reviewer 

\subsection*{Statement}
\textit{Hereby I state that submitted review is done objectively and indifferently and is solely based on professional, scientific and ethical standards.}

\raggedleft The reviewer.
%----------------------------------------------------------------------------------------

\end{document}